\documentclass[UTF8]{article}
\usepackage{ctex}
\usepackage{amsmath}
\usepackage{amssymb}
\author {林小斌}
\title {Problem Set 2}
\begin{document}
\maketitle

\section{Logistic Regression}
\begin{align*}
H_{i,j} &= \frac{\partial^2J(\theta)}{\partial \theta_i \partial \theta_j} \\
&= \frac{\partial}{\partial\theta_i}\left(-\frac{1}{m}\sum_{t=1}^m  \frac{1}{h_\theta(y^{(t)}x^{(t)})} \frac{\partial}{\partial\theta_j }h_\theta(y^{(t)}x^{(t)})\right) \\
&= \frac{\partial}{\partial\theta_i}\left(-\frac{1}{m}\sum_{t=1}^m  \frac{1}{h_\theta(y^{(t)}x^{(t)})}h_\theta(y^{(t)}x^{(t)})(1-h_\theta(y^{(t)}x^{(t)})) \frac{\partial}{\partial\theta_j }y^{(t)}\theta^Tx^{(t)}\right) \\
&= \frac{\partial}{\partial\theta_i}\left(-\frac{1}{m}\sum_{t=1}^m  (1-h_\theta(y^{(t)}x^{(t)}))y^{(t)}x_j^{(t)} \right) \\
&= \frac{1}{m}\sum_{t=1}^m\frac{\partial}{\partial\theta_i}\left(  h_\theta(y^{(t)}x^{(t)} \right)y^{(t)}x_j^{(t)} \\
&= \frac{1}{m}\sum_{t=1}^my^{(t)}x_j^{(t)}h_\theta\left(y^{(t)}x^{(t)}\right)\left(1-h_\theta(y^{(t)}x^{(t)})\right) \frac{\partial}{\partial\theta_i }y^{(t)}\theta^Tx^{(t)} \\
&= \frac{1}{m} \sum_{t=1}^m \left(y^{(t)}\right)^2 x_i^{(t)}x_j^{(t)} h_\theta\left(y^{(t)}x^{(t)}\right)\left(1-h_\theta(y^{(t)}x^{(t)})\right) \\
&= \frac{1}{m} \sum_{t=1}^m x_i^{(t)}x_j^{(t)} h_\theta\left(y^{(t)}x^{(t)}\right)\left(1-h_\theta(y^{(t)}x^{(t)})\right)
\end{align*}
thus
$$
H = \frac{1}{m}\sum_{t=1}^m\left[
h_\theta\left(y^{(t)}x^{(t)}\right)\left(1-h_\theta \left(y^{(t)}x^{(t)}\right)\right)x^{(t)}\left(x^{(t)}\right)^T
\right]
$$
where $x^{(t)} \in R^{n+1}$ and $x^{(t)}\left(x^{(t)}\right)^T \in R^{n+1 \times n+1}$.\\
Consider $z \in R^{n+1}$, we get the following formula:
$$
z^THz = \frac{1}{m}\sum_{t=1}^m\left[
h_\theta\left(y^{(t)}x^{(t)}\right)\left(1-h_\theta \left(y^{(t)}x^{(t)}\right)\right)z^Tx^{(t)}\left(x^{(t)}\right)^Tz
\right]
$$
(1)Consider $g(z)$ is sigmod function, then
$$h_\theta\left(y^{(t)}x^{(t)}\right)\left(1-h_\theta \left(y^{(t)}x^{(t)}\right)\right) > 0 $$
(2)Calculate $z^Tx^{(t)}\left(x^{(t)}\right)^Tz$
\begin{align*}
  z^Tx^{(t)}\left(x^{(t)}\right)^Tz
  &= \left(z^Tx^{(t)}\right)\left(\left(x^{(t)}\right)^Tz\right)\\
  &=
  \left(
  \left[\begin{matrix}
    z_1 & \dots & z_{n+1}
  \end{matrix}\right]
    \left[
     \begin{matrix}
      x_1^{(t)} \\
        \vdots \\
        x_{n+1}^{(t)}
     \end{matrix}
    \right]
  \right)
  \left(
  \left[\begin{matrix}
    x_1^{(t)} & \dots & x_{n+1}^{(t)}
  \end{matrix}\right]
    \left[
     \begin{matrix}
      z_1 \\
      \vdots \\
      z_{n+1}
     \end{matrix}
    \right]
  \right) \\
&=\left(z^Tx^{(t)}\right)^2 \geq 0
\end{align*}
In summary, $z^THz \geq 0$

\section{Regularized Normal Equation for Linear Regression}
\noindent Consider the cost function
\begin{align*}
  J(\theta)=\frac{1}{m} \left[
  \sum_{i=1}^m(h_\theta(x^{(i)}-y^{(i)})^2 - \lambda\sum_{j=1}^m\theta_j^2)
  \right]
\end{align*}
The normal equation is to find the parameters that minimize the cost function by solving the following equations.
\begin{equation*}
  \frac{\partial}{\partial\theta_j}J(\theta) = 0
\end{equation*}
Assuming that there are m training examples, each instance has n characteristics, the training example set is
\begin{equation*}
  X = \left[
  \begin{matrix}
      x_0^{(1)} & \dots &x_n^{(1)} \\
      \vdots & \ddots & \vdots \\
      x_0^{(m)} & \dots & x_n^{(m)}
  \end{matrix}
\right]
\end{equation*}
where $x_j^{(i)}$ represents the j feature of the i instance.\\
Consider
\begin{equation*}
  \theta = \left[
  \begin{matrix}
    \theta_0 & \theta_1 & \dots & \theta_n
  \end{matrix}
  \right]^T
\end{equation*}
\begin{equation*}
  Y = \left[
  \begin{matrix}
  y^{(1)} & y^{(2)} & \dots & y^{(m)}
  \end{matrix}
  \right]^T
\end{equation*}
thus
\begin{align*}
  J(\theta) &= \frac{1}{2m} \left[
   (X\theta - Y)^T(X\theta - Y) + \lambda\theta^2
  \right] \\
  &= \frac{1}{2m} \left[
   Y^TY - Y^TX\theta - \theta^TX^TY + \theta^TX^TX\thtea^T\theta + \lambda L\theta^2
  \right]
\end{align*}
where $L$ is $m \times m$ matrix and
$
  L = \left[
  \begin{matrix}
    0 &   &  & \\
      & 1 &  & \\
      &   & \ddots & \\
      &   &  & 1 &
  \end{matrix}
  \right]
$ \\ \\
Derivation is equivalent to the following form
\begin{equation*}
  \frac{1}{2m}\left(
  \frac{\partial Y^TY}{\partial \theta} - \frac{\partial Y^TX\theta}{\partial\theta} - \frac{\partial\theta^TX^TY}{\partial \theta} + \frac{\partial\theta^TX^TX\thtea^T\theta}{\partial\theta} + \lambda L\frac{\partial\theta^2}{\partial\theta}\right)
\end{equation*}

\paragraph{(1)For the first item}
$$\frac{\partial Y^TY}{\partial\theta} = 0$$
\paragraph{(2)For the second item}
\begin{align*}
  Y^TX\theta &=
  \left[
  \begin{matrix}
  y^{(1)} & y^{(2)} & \dots & y^{(m)}
  \end{matrix}
  \right]
  \left[
  \begin{matrix}
      x_0^{(1)} & \dots &x_n^{(1)} \\
      \vdots & \ddots & \vdots \\
      x_0^{(m)} & \dots & x_n^{(m)}
  \end{matrix}
 \right]
  \left[
  \begin{matrix}
    \theta_0 & \theta_1 & \dots & \theta_n
  \end{matrix}
  \right]^T \\
  &= \left(x_0^{(1)}y^{(1)} + \dots x_0^{(m)}y^{(m)} \right)\theta_0
  + \dots + \left(x_n^{(1)}y^{(1)} + \dots x_n^{(m)}y^{(m)} \right)\theta_n
\end{align*}
thus
$$
  \frac{\partial Y^TX\theta}{\partial\theta} = \left[
  \begin{matrix}
    \frac{\partial Y^TX\theta}{\partial\theta_0} \\
    \frac{\partial Y^TX\theta}{\partial\theta_1} \\
    \vdots \\
    \frac{\partial Y^TX\theta}{\partial\theta_n}
  \end{matrix}
  \right] = X^TY
$$

\paragraph{(3)For the third item}
\begin{align*}
  \theta^T X^TY &=
  \left[
  \begin{matrix}
    \theta_0 & \theta_1 & \dots & \theta_n
  \end{matrix}
  \right]
  \left[
  \begin{matrix}
      x_0^{(1)} & \dots &x_n^{(1)} \\
      \vdots & \ddots & \vdots \\
      x_0^{(m)} & \dots & x_n^{(m)}
  \end{matrix}
 \right]^T
 \left[
 \begin{matrix}
 y^{(1)} & y^{(2)} & \dots & y^{(m)}
 \end{matrix}
 \right]^T \\
  &= \left(x_0^{(1)}theta_0 + \dots x_0^{(m)}\theta_n \right)y^{(1)}
  + \dots + \left(x_n^{(1)}\theta_0 + \dots x_n^{(m)}\theta_n \right)y^{(n)}
\end{align*}
thus
$$
\frac{\partial \theta^T X^TY}{\partial\theta} = \left[
\begin{matrix}
  \frac{\partial \theta^T X^TY}{\partial\theta_0} \\
  \frac{\partial \theta^T X^TY}{\partial\theta_1} \\
  \vdots \\
  \frac{\partial \theta^T X^TY}{\partial\theta_n}
\end{matrix}
\right] = X^TY
$$

\paragraph{(4)For the fourth item}
$$
\theta^TX^TX\theta = X^TX\left( \theta_0^2 + \theta_1^2 +\cdots + \theta_n^2 \right)
$$
thus
$$
\frac{\partial \theta^TX^TX\theta}{\partial\theta} = \left[
\begin{matrix}
  \frac{\partial \theta^TX^TX\theta}{\partial\theta_0} \\
  \frac{\partial \theta^TX^TX\theta}{\partial\theta_1} \\
  \vdots \\
  \frac{\partial \theta^TX^TX\theta}{\partial\theta_n}
\end{matrix}
\right] = 2\left( X^TX\right)
\left[
  \begin{matrix}
    \theta_0 \\
    \theta_1 \\
    \vdots \\
    \theta_n
  \end{matrix}}
\right] = 2X^TX\theta
$$

\paragraph{(5)For the fifth item}
$$\lambda L \frac{\partial \theta^2}{\theta} = 2\lambda L \theta$$

\paragraph{In summary, the normal equation is:}
$$\frac{1}{2m}\left(-2X^TY+2X^TX\theta+2\lambda L\theta \right) = 0$$
thus
$$\theta = \left(X^TX + \lambda L \right)^{-1}X^TY$$


\section{Gaussian Discriminant Analysis Model}
\noindent According to the subject
\begin{align*}
  l(\psi,\mu_0,\mu_1,\Sigma) &= log\prod_{i=1}^m p(x^{(i)},y^{(i)};\psi,\mu_0,\mu_1,\Sigma) \\
  &=log\prod_{i=1}^m p(x^{(i)}|y^{(i)};\psi,\mu_0,\mu_1,\Sigma)p(y^{(i)};\psi) \\
  &=\sum_{i=1}^mlogp(x^{(i)}|y^{(i)};\mu_0,\mu_1,\Sigma) +
  \sum_{i=1}^mlogp(y^{(i)};\psi) \\
  &=\sum_{i=1}^mlogp(x^{(i)}|y^{(i)}=0)^{1-y^{(i)}} \cdot p(x^{(i)}|y^{(i)}=1)^{y^{(i)}} + \sum_{i=1}^mlogp(y^{(i)}) \\
  &=\sum_{i=1}^m(1-y^{(i)})logp(x^{(i)}|y^{(i)}=0) + \sum_{i=1}^my^{(i)}logp(x^{(i)}|y^{(i)}=1) + \sum_{i=1}^mlogp(y^{(i)})
\end{align*}
\paragraph{(1)Finding partial derivatives for $\psi$}

\begin{align*}
\frac{\partial l(\psi,\mu_0,\mu_1,\Sigma)}{\partial \psi}
&= \frac{\sum_{i=1}^mlogp(y^{(i)})}{\partial \psi} \\
&= \frac{\partial \sum_{i=1}^m \psi^{y^{(i)}}(1-\psi)^{1-y^{(i)}}}{\partial\psi} \\
&= \frac{\partial \sum_{i=1}^m \left(
y^{(i)}log\psi +(1-y^{(i)})log(1-\psi)
\right)}{\partial\psi} \\
&= \sum_{i=1}^m\left(
y^{(i)}\frac{1}{\psi} - (1-y^{(i)})\frac{1}{1-\psi}\right) \\
&= \sum_{i=1}^m\left(
I(y^{(i)}=1)\frac{1}{\psi} - I(y^{(i)}=0)\frac{1}{1-\psi}
\right)
\end{align*}
where $I$ is Indicator function, let the formula be zero, we get the final $\psi$:
$$
 \psi = \frac{\sum_{i=1}^mI(y^{(i)}=1)}{\sum_{i=1}^m\left(
I(y^{(i)}=0) + I(y^{(i)}=1)\right)} = \frac{\sum_{i=1}^mI(y^{(i)}=1)}{m}
$$

\paragraph{(2)Finding partial derivatives for $\mu_0$ and $\mu_1$}
\begin{align*}
  \frac{\partial l(\psi,\mu_0,\mu_1,\Sigma)}{\partial \mu_0}
  &=\frac{\partial(1-y^{(i)})logp(x^{(i)}|y^{(i)}=0)}{\partial \mu_0}\\
  &=\frac{\sum_{i=1}^m(1-y^{(i)})\left(log\frac{1}{(2\pi)^{n/2}|\Sigma|^{1/2}}-\frac{1}{2}(x^{(i)}-\mu_0)^T\Sigma^{-1}(x^{(i)}-\mu_0)\right)}{\partial \mu_0} \\
  &=\sum_{i=1}^m(1-y^{(i)})(\Sigma^{-1}(x^{(i)}-\mu_0)) \\
  &=I(y^{(i)}=0)\Sigma^{-1}(x^{(i)}-\mu_0)
\end{align*}
Let the formula be zero, we get the final $\mu_0$:
$$\mu_0 = \frac{\sum_{i=1}^mI(y^{(i)}=0)x^{(i)}}{\sum_{i=1}^mI(y^{(i)}=0)}
$$
According to symmetry
$$\mu_1 = \frac{\sum_{i=1}^mI(y^{(i)}=1)x^{(i)}}{\sum_{i=1}^mI(y^{(i)}=1)}
$$

\paragraph{(3)Finding partial derivatives for $\Sigma$}
The following is a partial derivative of $\Sigma$. Since only the first two parts of the likelihood function are related to $\Sigma$, the first two parts are rewritten as follows
\begin{align*}
  &\sum_{i=1}^m(1-y^{(i)})logp(x^{(i)}|y^{(i)}=0) + \sum_{i=1}^my^{(i)}logp(x^{(i)}|y^{(i)}=1) \\
  &=\sum_{i=1}^m(1-y^{(i)})\left(log\frac{1}{(2\pi)^{n/2}|\Sigma|^{1/2}}-\frac{1}{2}(x^{(i)}-\mu_0)^T\Sigma^{-1}(x^{(i)}-\mu_0)\right) \\
& \qquad+ \sum_{i=1}^my^{(i)}\left(log\frac{1}{(2\pi)^{n/2}|\Sigma|^{1/2}}-\frac{1}{2}(x^{(i)}-\mu_1)^T\Sigma^{-1}(x^{(i)}-\mu_1)\right) \\
&=\sum_{i=1}^m\left(log\frac{1}{(2\pi)^{n/2}|\Sigma|^{1/2}}-\frac{1}{2}(x^{(i)}-\mu_{y^{(i)}})^T\Sigma^{-1}(x^{(i)}-\mu_{y^{(i)}})  \right)\\
&=\sum_{i=1}^m\left(-\frac{n}{2}log(2\pi)-\frac{1}{2}log(|\Sigma|)\right)-\frac{1}{2}\sum_{i=1}^m(x^{(i)}-\mu_{y^{(i)}})^T\Sigma^{-1}(x^{(i)}-\mu_{y^{(i)}})
\end{align*}
thus
\begin{align*}
  \frac{\partial l(\psi,\mu_0,\mu_1,\Sigma)}{\partial \Sigma} &=
-\frac{1}{2}\sum_{i=1}^m(\frac{1}{|\Sigma|}|\Sigma|\Sigma^{-1})-\frac{1}{2}\sum_{i=1}^m(x^{(i)}-\mu_{y^{(i)}})(x^{(i)}-\mu_{y^{(i)}})^T\frac{\partial\Sigma^{-1}}{\partial\Sigma} \\
&=-\frac{m}{2}-\frac{1}{2}\sum_{i=1}^m(x^{(i)}-\mu_{y^{(i)}})(x^{(i)}-\mu_{y^{(i)}})^T(-\Sigma^{-2})
\end{align*}
The following formula is used for derivation.
$$
\frac{\partial|\Sigma|}{\partial\Sigma}=|\Sigma|\Sigma^{-1},
\frac{\partial\Sigma^{-1}}{\partial\Sigma}=-\Sigma^{-2}
$$
Let the formula be zero, we get the final $\Sigma$:
$$
\Sigma=\frac{1}{m}\sum_{i=1}^m(x^{(i)}-u_{y^{(i)}})(x^{(i)}-u_{y^{(i)}})^T
$$

\section{MLE for Naive Bayes}
\paragraph{(i)}
According to the question,
$$
  p^*=arg\ max_{p \in P_y} \sum_{y\in Y}c_ylog p_y
$$
limited to $\sum_{y \in Y}p_y = 1$.
Using Lagrange multiplier method, we have
$$
L(p,\lambda) = \sum_{y\in Y}c_ylog p_y + \lambda(\sum_{y \in Y}p_y-1)
$$
The derivation of $p_1,p_2,\dots,p_y$ is 0 respectively.

\begin{align*}
&  \frac{c_1}{p_1} + \lambda = 0 \\
&  \frac{c_2}{p_2} + \lambda = 0 \\
&  \vdots  \\
&  \frac{c_y}{p_y} + \lambda = 0 \\
&  \sum_{y \in Y}p_y = 1 \\
&  \sum_{y \in Y}c_y = N
\end{align*}
Above all, we have
$$ p_y^*=\frac{c_y}{N}$$

\paragraph{(ii)} Maximum-likelihood Estimates for Naive Bayes
\begin{align*}
  l(\Omega) &=\sum_{i=1}^mlog\ p(x^{(i)},y^{i}) \\
  &=\sum_{i=1}^mlog \left(p(y^{(i)})\prod_{j=1}^np_j(x_j^{(i)}|y^{(i)})\right)\\
  &=\sum_{i=1}^mlog\ p(y^{(i)}) +\sum_{i=1}^m\sum_{j=1}^nlog p_j(x_j^{(i)}|y^{(i)}) \\
  &= \sum_{y=1}^kcount(y)log\ p(y) + \sum_{j=1}^n\sum_{y=1}^k\sum_{x\in{0,1}}count_j(x|y)log\ p_j(x_j^{(i)}|y^{(i)})
\end{align*}
where
\begin{align*}
  &count(y) = \sum_{i=1}^mI(y^{(i)}=y),\forall y \in 1,2,\dots,k \\
  &count(x|y) = \sum_{i=1}^mI(y^{(i)}=y,x_j^{(i)}=x),\forall y \in 1,2,\dots,k,\forall x \in 0,1
\end{align*}
The $\frac{\partial l(\Omega)}{\partial p(y)}$ is not related to the second one. Using Lagrange multiplier method, we have
$$L_1(\Omega,\lambda_1) = \sum_{y=1}^kcount(y)log\ p(y)+\lambda_1(\sum_{y=1}^k\ p(y)-1)$$
where $\sum_{y=1}^k\ p(y)=1$. Thus
\begin{align*}
&\frac{\partial L_1(\Omega,\lambda_1)}{\partial p(y)} =\frac{count(y)}{p(y)} + \lambda_1= 0 \\
&p(y) = \frac{-count(y)}{\lambda_1} \\
&\sum_{y=1}^k\ p(y) = -\frac{1}{\lambda_1}\sum_{y=1}^kcount(y) = 1
\end{align*}
Then we have
\begin{align*}
&\lambda_1 = - \sum_{y=1}^kcount(y) = -m \\
  &p(y) = \frac{count(y)}{m} = \frac{\sum_{i=1}^mI(y^{(i)}=y)}{m} \tag{$1$}
\end{align*}
Similarity, we have
$$L_2(\Omega,\lambda_2) = \sum_{j=1}^n\sum_{y=1}^k\sum_{x\in 0,1}count_j(x|y)log\ p_j(x|y)+\lambda_2(\sum_{x\in 0,1}\ p(x|y)-1)$$
where $\sum_{x\in 0,1}\ p(x|y)$. Thus

\begin{align*}
&\frac{\partial L_2(\Omega,\lambda_2)}{\partial p_j(x|y)} =\frac{count_j(x|y)}{p_j(x|y)} + \lambda_2= 0 \\
&p_j(x|y) = \frac{-count_j(x|y)}{\lambda_2} \\
&\sum_{x\in 0,1}\ p(x|y) = -\frac{1}{\lambda_2}\sum_{x\in 0,1}count_j(x|y) = 1\\
&\lambda_2 = -\sum_{x\in 0,1}count_j(x|y) = -\sum_{i=1}^mI(y^{(i)}=y)
\end{align*}
Then we have
\begin{align*}
  &p_j(x|y) = \frac{-count_j(x|y)}{-\sum_{i=1}^mI(y^{(i)}=y)} = \frac{\sum_{i=1}^mI(y^{(i)}=y,x_j^{(i)}=x)}{\sum_{i=1}^mI(y^{(i)}=y)} \tag{$2$}
\end{align*}

\end{document}
