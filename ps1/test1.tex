\documentclass[UTF8]{article}
\usepackage{ctex}
\usepackage{amsmath}
\usepackage{amssymb}
\author {林小斌}
\title {Problem Set 1}
\begin{document}
\maketitle
\section{Conditions for Normal Equation}
\quad If the matrix $A^TA$ is invertible,matrix $(A^TA)^{-1}$ satisfy the equation
$$
A^TA \cdot (A^TA)^{-1} =  (A^TA)^{-1} \cdot A^TA = I
$$
where $A$ is $n \times m$ matirx, $A^TA$ is $m \times m$ matrix.,$I$ is unit matrix.
According to the characteristics of the invertible matrix,The equation $A^TAx= 0$ has only one normal solution.
$$
A^TAx = 0  \Longrightarrow  x^TA^TAx = 0 \Longrightarrow (Ax)^TAx =0 \Longrightarrow \|  Ax \|^2 =0
$$
Then $Ax = 0$,where $x = 0$. So the columns of A are linearly independent.

\quad if the columns of A are linearly independent,then $Ax = 0$ has only one normal solution $x = 0$.   So the null space of $A^TAx=\{0\}$. Since $A^TA$ is a square matrix, this means $A^TA$ is invertible.

\section{Newton's Method for Computing Least Squares}
\paragraph{(a)}
\begin{align*}
H_{i,j} &= \frac{\partial^2J(\theta)}{\partial \theta_i \partial \theta_j} \\
&=  \sum_{t=1}^m \frac{\partial}{\partial \theta_j} (\theta^Tx^{(t)}-y^{(i)})x_i^{(t)} \\
&=  \sum_{t=1}^m x_i^{(t)} \frac{\partial}{\partial \theta_j} \theta^Tx^{(t)} \\
&= \sum_{t=1}^m x_i^{(t)}x_j^{(t)}
\end{align*}
thus

\begin{align*}
H &= \sum_{t=1}^m
\begin{pmatrix}
        x_1^{(t)}x_1^{(t)} &  x_1^{(t)}x_2^{(t)} & \cdots &  x_1^{(t)}x_{n+1}^{(t)} \\
        x_2^{(t)}x_1^{(t)} &  x_2^{(t)}x_2^{(t)} &  \cdots &  x_2^{(t)}x_{n+1}^{(t)} \\
       \vdots & \vdots  & \ddots & \vdots \\
        x_m^{(t)}x_1^{(t)} &  x_m^{(t)}x_2^{(t)} & \cdots &  x_m^{(t)}x_{n+1}^{(t)}
\end{pmatrix}
& = \sum_{t=1}^m x^{(i)}(x^{(i)})^T \tag{$2.1$}
\end{align*}
where $x^{(i)} \in R^{n+1}$.
\paragraph{(b)}
According to Newton's method, we have the following update \\
\begin{equation}
 \theta^{t+1} = \theta^t - H^{-1}\bigtriangledown_\theta J(\theta) \tag{$2.2$}
\end{equation}
 Before iteration,suppose $\theta^t=0$,after first iteration,we get
$$
   \theta = -H^{-1}\bigtriangledown_\theta J(\theta) \\
   = -H^{-1}\sum_{t=1}^m (\theta^Tx^{(t)}-y^{(i)})x_i^{(t)}
$$ \\
\\
\large Let
\begin{equation}
  X =
\left[
\begin{array}{c}
 (x^{(1)})^T  \\
 (x^{(2)})^T  \\
 \vdots \\
 (x^{(m)})^T
\end{array}
\right]  \tag{$2.3$}
\end{equation}
where $x^{(i)}=[x_i^{(1)} \  x_i^{(2)} \ \dots \ x_i^{(n+1)} ]^T$.
Caculate $X^TX$ and $X^T \vec{y}$,then we get \\
\begin{equation}
  X^TX =
  \left[
  \begin{array}{cccc}
   x^{(1)} \ x^{(2)} \ \dots \ x^{(m)}
  \end{array}
  \right]
  \left[
  \begin{array}{c}
   (x^{(1)})^T  \\
   (x^{(2)})^T  \\
   \vdots \\
   (x^{(m)})^T
  \end{array}
  \right]
  = \sum_{t=1}^m x^{(i)}(x^{(i)})^T
  = H
\tag{$2.4$}
\end{equation}
\begin{equation}
  X^T \vec{y} =
  \left[
  \begin{array}{cccc}
   x^{(1)} \ x^{(2)} \ \dots \ x^{(m)}
  \end{array}
  \right]
    \left[
    \begin{array}{c}
     (y^{(1)}) \\
     (y^{(2)})  \\
     \vdots \\
     (y^{(m)})
    \end{array}
      \right]
      = \sum_{t=1}^mx^{(i)}y^{(i)}
    \tag{$2.5$}
\end{equation}
And
\begin{equation}
  \sum_{t=1}^m (\theta^Tx^{(t)}-y^{(i)})x_i^{(t)} =
  \sum_{t=1}^m -y^{(i)}x_i^{(t)} = -X^T \vec{y}
  \tag{$2.6$}
\end{equation}
Thus
$$
\theta^* = \theta = - H^{-1}\bigtriangledown_\theta J(\theta) =
(X^TX)^{-1}X^T \vec{y}
$$
\section{Prediction using Linear Regression}
\paragraph{(a)}In linear regression, the least squares method is to try to find a straight line that minimizes the sum of Euclidean distances of all samples onto a straight line. \\
\large {\textbf{Hypothesis}}:

\begin{equation}
  E_{(a,b)}= \sum_{i=1}^m(y_i-ax_i-b)^2 \tag{$3.1$} \\
\end{equation} \\
Deriving $a$ and $b$ through $E_{(a,b)}$: \\
\begin{align*}
&\frac{\partial E_{(a,b)}}{\partial a}=2(a \sum_{i=1}^m x_i^2-\sum_{i=1}^m(y_i-b)x_i) \tag{$3.2$}\\
&\frac{\partial E_{(a,b)}}{\partial b}=2(mb-\sum_{i=1}^m(y_i-ax_i)) \tag{$3.3$}
\end{align*}
Let the formulas (3.2) and (3.3) be zero to get the close-form of a and b optimal solutions. \\
\begin{align*}
&a = \frac{\sum_{i=1}^my_i(x_i-\bar{x})}{\sum_{i=1}^m x_i^2-\frac{1}{m}(\sum_{i=1}^m x_i)^2} \tag{$3.4$} \\
&b =\frac{1}{m}\sum_{i=1}^m(y_i - ax_i) \tag{$3.5$}
\end{align*} \\
where $\bar{x} = \frac{1}{m}\sum_{i=1}^m x_i$. \\ \\
\large {\textbf{Solution}}: \\
\begin{align*}
& \bar{x} = \frac{2005+2006+2007+2008+2009}{5} = 2007 \\
& a = \frac{12 \times (-2) + 19 \times (-1) + 29 \times 0 + 37 \times 1 + 45 \times 2}{(2005^2+2006^2+2007^2+2008^2+2009^2)-\frac{1}{5}(2005+2006+2007+2008+2009)^2} \\
&= 8.4 \\
&b = \frac{1}{5}[(12-16842)+(19-16850.4)+(29- 16858.8)+(37 - 16867.2) +(45 - 16875.6)] \\
&=-16830.4
\end{align*}
The resulting linear equation is
\begin{equation}
  y = 8.4x - 16830.4 \tag{$3.6$}
\end{equation}

\paragraph{(b)}
Substituting $x=2012$ into equation 3.6,
$$
y = 2012 \times 8.4 - 16830.4 = 70.4 (\text{million dollars})
$$

\end{document}
